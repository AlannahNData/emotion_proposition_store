% Chapter 1

\chapter{Introduction} % Chapter title

\label{ch:introduction} % For referencing the chapter elsewhere, use \autoref{ch:introduction} 

%----------------------------------------------------------------------------------------

BA thesis - outline

\section{Sources for emotion-triggering patterns}

Don't use ambiguous ones (e.g. anxious can mean both eager and worried)
Intuitively, we would want to employ mainly transitive constructions in which the subject denotes the experiencer and the object refers to the cause, also constructions in which subject is experiencer and clause denotes object



\subsection{Introspection}



\subsection{Emotions in VerbNet, FrameNet}
VerbNet is not really useful because we don't care about particular verbs, but emotions and propositions. VerbNet doesn't differentiate between different emotions only between different alternations, e.g. "fear" is a member of the "admire" class.

--> marvel at?

FrameNet has frames for
fear
lexical units that create this frame
afraid.a, apprehension.n, dread.n, fear.n, freaked.a, frightened.a, $live_in_fear$.v, nervous.a, scared.a, terrified.a, terror.n

trust
lexical units: believe.v, credence.n, credulous.a, faith.n, gullible.a, reliability.n, reliable.a, trust.n, trust.v, trustworthy.a

$cause_emotion$
lexical units: affront.n, affront.v, $call_names$.v, concern.v, insult.n, insult.v, offend.v, offense.n, offensive.a



\subsection{Emotion verb classes from Mathieu and Fellbaum}
A corpus-based construction of emotion verb classes
reread 1. Introduction and 2. Emotion verbs

surprise: astonish, surprise, amaze, astound, strike, stun, floor, dumbfounded, flabbergasted, stupefy
fear: intimidate, scare, frighten, alarm, terrify



\subsection{Dictionaries and thesauri}

Oxford English Dictionary
Most of the time, first sense was taken
\begin{itemize}
	\item Joy: \textit{A vivid emotion of pleasure arising from a sense of well-being or satisfaction; the feeling or state of being highly pleased or delighted; exultation of spirit; gladness, delight.}
	\item Trust: \textit{Confidence in or reliance on some quality or attribute of a person or thing, or the truth of a statement. Const. in (of, on, upon, to, unto).}
	\item Fear: \textit{The emotion of pain or uneasiness caused by the sense of impending danger, or by the prospect of some possible evil.}
	\item Surprise: \textit{The feeling or mental state, akin to astonishment and wonder, caused by an unexpected occurrence or circumstance.} first sense is act of surprise, military act, etc.
	\item Sadness: \textit{The condition or quality of being sad (Of a person, or his or her feelings, disposition, etc.: feeling sorrow; sorrowful, mournful, heavy-hearted.).} Obsolete senses precede it.
	\item Disgust: \textit{Strong repugnance, aversion, or repulsion excited by that which is loathsome or offensive, as a foul smell, disagreeable person or action, disappointed ambition, etc.; profound instinctive dislike or dissatisfaction.}
	\item Anger: \textit{The active feeling provoked against the agent; passion, rage; wrath, ire, hot displeasure.}
	\item Anticipation: \textit{The action of looking forward to, expectation.}
\end{itemize}
joy	be satisfied that
joy	be pleased that
joy	be delighted that
joy	be glad that
sadness	feel sorrow that
sadness mourn
disgust	loathe
disgust dislike
disgust be repulsed by
disgust be aversed to
anger	be enraged
anger	be wrathful
anger	be irate
anticipation	look forward to
anticipation expect


The Free Dictionary
joy	enjoy
joy	take pleasure in
joy	be content
joy	rejoice
fear	be uneasy about
fear	be apprehensive about
fear	dread
trust	rely on
trust	have/place reliance
trust	depend
sad	be melancholic that


Merriam-Webster
synonyms for verbs
anger: enrage, incense, inflame (also enflame), infuriate, ire, madden, outrage, rankle, rile, roil, steam up, tick off
fear: bother, fear, fret, fuss, stew, stress, sweat, trouble
joy (rejoice): crow, delight, exuberate, glory, jubilate, joy, kvell, rejoice, triumph
sadness (sadden): bum (out), burden, dash, deject, get down, oppress, sadden, weigh down
anticipation (anticipate): anticipate, await, hope (for), watch (for)
disgust: gross out, nauseate, put off, repel, repulse, revolt, sicken, turn off

Note: allow for one modifier of adjective


Roget's Thesaurus (thesaurus.com)
as used for target terms in Crowdsourcing a Word-Emotion Association Lexicon
chosen verbs with highest relevance
anticipate: expect, predict, assume, await, count on, forecast, foresee, prepare for, see
anger: aggravate, annoy, antagonize, arouse, displease, embitter, enrage, exacerbate, exasperate, excite, incense, inflame, infuriate, irritate, offend, outrage, provoke, rankle, rile
fear: feel alarmed, be scared off, anticipate, avoid, dread, expect, foresee, shun, suspect, worry
joy: exult, revel, make happy, delight, amuse, attract, charm, cheer, enchant, enrapture, entertain, fascinate, gratify, please, rejoice, satisfy, thrill, wow
sadness (sadden): discourage, dishearten, dispirit, grieve
disgust: bother, disenchant, displease, disturb, insult, irk, nauseate, offend, outrage, revolt, shock, sicken, turn off, upset
surprise: astonish, amaze, astound, awe, bewilder, confound, confuse, dazzle, disconcert, dismay, dumbfound, flabbergast, overwhelm, perplex, rattle, shock, startle, stun, unsettle
trust: count on, depend on, look to

write program that converts list such as:
joy	be happy (that) S

to regex format


\subsection{Sentiment lexica (highly associated verbs)}

Harvard General Inquirer
Pstv 1045 positive words, an earlier version of Positiv.
A subset of 557 words are also tagged Affil for words indicating affiliation or supportiveness.
Ngtv 1160 negative words, an earlier version of Negativ.
A subset of 833 words are also tagged Hostile for words indicating an attitude or concern with hostility or aggressiveness.
Pleasur168 words indicating the enjoyment of a feeling, including words indicating confidence, interest and commitment.
Pain 254 words indicating suffering, lack of confidence, or commitment.
Feel 49 words describing particular feelings, including gratitude, apathy, and optimism, not those of pain or pleasure.
-- not really ordered, useful for our purposes
Arousal 166 words indicating excitation, aside from pleasures or pains, but including arousal of affiliation and hostility.
EMOT 311 words related to emotion that are used as a disambiguation category, but also available for general use.
--> have notes for some adjective: http://www.wjh.harvard.edu/~inquirer/EMOT.html
joy; admire; adore; appreciate; be fond of
trust
fear
surprise
sadness
disgust
anger
anticipation

WordAffectLexicon



EmoLex (crowd-sourced word-emotion association lexicon)


SentiWordNet
Helpful, doesn't yield any additional insights, though


\subsection{Adjectives}

- Adjectives are more indicate of emotion than nouns or verbs (see also what Anette said)

- In WordNet, nouns and verbs are clustered in synsets and supersenses. No taxonomic hierarchy for adjectives. \cite{adjective_supersenses} induce supersenses for adjectives taking GermaNet's guidelines\footnote{http://www.sfs.uni-tuebingen.de/lsd/adjectives.shtml} as inspiration.

- They build a weakly supervised classifier that labels adjective types (irrespective of context), which they train on a small set of seed examples, some of them translations of GermaNet. We take the translations that pertain to FEELING and manually label them with Plutchik's 8 emotion classes. Furthermore, they released 7511 WordNet adjectives tagged by their classifier with a an adjective type vector. From these we derive the XXX adjectives that are labeled with the adjective type FEELING; we take the label as indicative of the emotion, as the classifier has a high accuracy (k-4 accuracy of 91\%).


\section{Compilation}

From the previously mentioned sources, we derive now the patterns. We end up with the following initial patterns:

trust	count on	NP	false
trust	depend on	NP	false
trust	look to	NP	false
trust	trust (in)	NP	false
trust	trust that	S	false
trust	rely on	NP	false
trust	place reliance on	NP	false

anticipation	expect	NP	false
anticipation	expect that	S	false
anticipation	predict	NP	false
anticipation	predict that	S	false
anticipation	assume NP	false
anticipation	assume that	S	false
anticipation	await	NP	false
anticipation	await	that S	false
anticipation	count on	NP	false
anticipation	count on	S	false
anticipation	forecast	NP	false
anticipation	forecast that S	false
anticipation	foresee	NP	false
anticipation	foresee that	S	false
anticipation	prepare for	NP	false
anticipation	prepare for	S	false
anticipation	see	NP	false
anticipation	see that	S	false
anticipation	anticipate	NP	false
anticipation	hope for	NP/S	false
anticipation	look forward to	NP/S	false


anger	aggravate	NP	true
anger	annoy	NP	true
anger	antagonize	NP	true
anger	arouse	NP	true
anger	displease	NP	true
anger	embitter	NP	true
anger	enrage	NP	true
anger	exacerbate	NP	true
anger	exasperate	NP	true
anger	excite	NP	true
anger	incense	NP	true
anger	inflame	NP	true
anger	infuriate	NP	true
anger	irritate		NP	true
anger	offend	NP	true
anger	outrage	NP	true
anger	provoke	NP	true
anger	rankle	NP	true
anger	rile	NP	true
anger	be angry that	S	false
anger	be wrathful that	S	false
anger	be irate that	S	false

fear	alarm	NP	true
fear	scare	NP	true
fear	anticipate	NP	false
fear	avoid	NP	false
fear	dread	NP	false
fear	expect	NP	false
fear	foresee	NP	false
fear	shun	NP	false
fear	suspect	NP	false
fear	worry	NP	true
fear	fear	NP	false
fear	fear that	S	false
fear	be afraid of	NP	false
fear	be afraid that	S	false
fear	be apprehensive about	S	false
fear	frighten	NP	true
fear	be frightened by	NP	false
fear	be frightened that	S	false
fear	scare	NP	true
fear	be scared of	NP	false
fear	be scared that	S	false
fear	spook	NP	true
fear	terrify	NP	true
fear	petrify	NP	true
fear	intimidate	NP	true
fear	alarm	NP	true

joy	revel in	NP	false
joy	delight	NP	true
joy	amuse	NP	true
joy	attract	NP	true
joy	charm	NP	true
joy	cheer	NP	true
joy	delight	NP	true
joy	enchant	NP	true
joy	enrapture	NP	true
joy	entertain	NP	true
joy	enjoy	NP/S	false
joy	take pleasure in	S	false
joy	fascinate	NP	true
joy	gratify	NP	true
joy	please	NP	true
joy	rejoice	that	S	false
joy	satisfy	NP	true
joy	thrill	NP	true
joy	wow	NP	true
joy	be happy (?!to) that	S
joy	be glad (?!to) that	S
joy	be content (?!to) that	S


sadness	sadden	NP	true
sadness	discourage	NP	true
sadness	dishearted	NP	true
sadness	dispirit	NP	true
sadness	grieve that S	false
sadness	be sad that	S	false
sadness	be melancholic that	S	false
sadness	feel sorrow that	S	false
sadness	mourn	NP/S	false

disgust	be averse to	NP	false
disgust	disgust	NP	true
disgust	bother	NP	true
disgust	disenchant	NP	true
disgust	dislike	NP	false
disgust	displease	NP	true
disgust	disturb	NP	true
disgust	insult	NP	true
disgust	irk	NP	true
disgust	loathe	NP	false
disgust	nauseate	NP	true
disgust	offend	NP	true
disgust	outrage	NP	true
disgust	put off	NP	true
disgust	repulse	NP	true
disgust	revolt	NP	true
disgust	shock	NP	true
disgust	sicken	NP	true
disgust	turn off	NP	true
disgust	upset	NP	true
disgust	hate	NP	false
disgust	hate that	S	false

surprise	astonish	NP	true
surprise	amaze	NP	true
surprise	astound	NP	true
surprise	awe	NP	true
surprise	bewilder	NP	true
surprise	confound	NP	true
surprise	confuse	NP	true
surprise	dazzle	NP	true
surprise	disconcert	NP	true
surprise	dismay	NP	true
surprise	dumbfound	NP	true
surprise	flabbergast	NP	true
surprise	overwhelm	NP	true
surprise	perplex	NP	true
surprise	rattle	NP	true
surprise	shock	NP	true
surprise	surprise	NP	true
surprise	startle	NP	true
surprise	stun	NP	true
surprise	unsettle	NP	true
suprise	stupify	NP	true

\section{Selection of corpus}

default choice: news corpus of Gigaword
not too rich in reporting emotions other than anger and expectation
genre could clearly play a role.
Both thematically (which is more domain than genre) and stylistically (this is what is really meant by genre)
e.g. if working on novels, we'd certainly get a wider variety of emotional expressions.

GW contains the NYT corpus that is annotated for domains. 
Eva Sourjikova analyzed this and extracted specific domain sub-corpora. 

!!Test whether applying your extractions on categories like: Opinion/Letters; MentalHealthDisorders; NY City; Child abuse and neglect; freedom and human rights, etc. are more prone for more varied emotions.

!!Pre-select domains that are emotion-prone by checking which of them have a high rate of emotion words from those that you pre-selected.

Outlook: turn to other corpora of different genres, e.g. ukWaC is richer in adjective meaning variety, this we know from Matthias Hartungs work). There is also a parsed version (pukWaC).

\section{General guidelines}

Our goal: Harvesting
it's not possible to take all, particularly complex/complicated constructions, e.g. multiple embeddings, proverbs, idionsyncracies.

take clear cases, easy structures that can be easily disambiguated

define filter for extraction, take only those patterns, whose proposition is a clear-cut form

recognize passive, normalize diathesis --> how?

in case of too few items, corpus can be bigger; 
--> statistics necessary of how big corpus should be

examples indicate that often that often modifiers, pre-modifiers, etc. are highly relevant for the sentiment
--> generally interesting to save all meaningful dependencies
--> only possible if propositions have a reasonable size
--> estimate length of simple embedded sentences, possibly with/without cutting off of adjuncts (not arguments)

lemma for representational level; also index advisable to be able to go back and check if further normalisations or elements have to be taken care of

part-of-speech labeling is important, as e.g. trust or fear can function both as verbs and nouns.


\section{Important observations}

- only had used Stanford present verb tag label before, inclusion of other labels increased recall from 1/400 to 1/100
- many matches involved the same trigger words; not bad by itself. If unambigious and frequent, this is exactly what we want.
- two interesting aspects related to this:
1: find frequent and unambiguous trigger contexts so that from these we can find indications on what are secondary emotion words in their scope (complements)
2: (which we did not choose as primary step to take if I recall correctly) is to acquire a wider variation of emotion indicating words. This we could still do in a second step.

- both corpus as well as indicating expressions matter (see above)

\subsection{Domain}

According to Eva Sourjikova:
Semantic content of a text / Subject
Domain is determined by the semantic subject of the text mesage. It can be classified under one
or more general topic and into a set of specific sub-topics or also a combination of more topics.
Some words (terms) are more likely to appear for a given domain, so called domain terms. As
a conceptual ontology, we can imagine a hierarchy of domain specific texts. Some possible tasks:
text classification, topic detection, controlled vocabulary. terms exhibit a degree of domain independence. Some terms are more likely to appear for a given topics, while other terms are far more
generic and will appear in almost every topic regardless of how similar the topics are

\section{Remaining questions}

Look at indicating expressions themselves.
Many of categories more related to adjectival or predicative expressions and less clearly related to verb constructions
--> useful to include constructions that involve adjectives, e.g. be glad that; be afraid that; ..

!!paper \citep{adjective_supersenses} induce supersenses for adjectives. One class is emotion. These should be very useful trigger words. And it would be interesting if one can classify them into these 8 emotion classes. 



\section{Next steps}

1. apply current method to some varied subgenres/domains (guided by the NTY categories) and plot the effects on the basis of the classified patterns you have at the moment

2. consider including more word categories.
adjectives: in most cases still able to identify bearer of emotion (ask for advise, if necessary)
nouns: nominalized verbs from NomBank; possibly more matches in standard text genres, more difficult to identify the bearer of the emotion; often implicit
adjectives are most promising

Anette:
When I looked at the sentence examples, and also the cut-down predicates of the embedded sentences, I was wondering whether extracting verb argument structures from the acquired propositions is the right way to go. Maybe it will be possible with a better filtering technique (getting more data - and there should be enough - but cleaning it by better filters). But we could, in a first step, also use the bare embedded material and apply topic modeling to them. This would mean we induce topics for embedded contexts that are pre-labelled with the embedding emotion indicating predicates. This way, we should be able to induce topics that reflect these emotions, and more filtering could be applied then, to sharpen these contexts or topics in better ways. This is related to the work I did with Matthias Hartung \cite{hartung2011exploring}. It's a kind of distant supervision for inducing semantically constrained LDA topics.

But also using some association statistics for detecting collocations could be a first step to see what we can get out of these contexts.


- Possibly more interesting to investigate nature/clarity of Plutchik's emotion classes in contrast to a pure pos/neg/neutral classification. 

- Ask if there is a better way to handle passive voice?








%This template for \LaTeX\ has two goals:
%\begin{enumerate}
%\item Provide students with an easy-to-use template for their Master's or PhD thesis (though it might also be used by other types of authors for reports, books, etc.).
%\item Provide a classic, high-quality typographic style that is inspired by \citeauthor{bringhurst:2002}'s ``\emph{The Elements of Typographic Style}'' \citep{bringhurst:2002}.
%\marginpar{\myTitle \myVersion}
%\end{enumerate}
%
%The bundle is configured to run with a \emph{full} MiK\TeX\ or \TeX Live installation right away and, therefore, it uses only freely available fonts.
%
%People interested only in the nice style and not the whole bundle can now use the style stand-alone via the file \texttt{classicthesis.sty}. This works now also with ``plain'' \LaTeX.
%
%As of version 3.0, \texttt{classicthesis} can also be easily used with \mLyX\footnote{\url{http://www.lyx.org}} thanks to Nicholas Mariette and Ivo Pletikosi\'c. The \mLyX\ version of this manual will contain more information on the details.
%
%This should enable anyone with a basic knowledge of \LaTeXe\ or \mLyX\ to produce beautiful documents without too much effort. In the end, this is my overall goal: more beautiful documents, especially theses, as I am tired of seeing so many ugly ones.
%
%The whole template and the used style is released under the \textsmaller{GNU} General Public License. 
%
%If you like the style then I would appreciate a postcard:
%\begin{center}
%Andre Miede \\
%Detmolder Strasse 32 \\
%31737 Rinteln \\
%Germany
%\end{center}
%
%\noindent The postcards I received so far are available at:
%\begin{center}
% \url{http://postcards.miede.de}
%\end{center}
%\marginpar{A well-balanced line width improves the legibility of the text. That's what typography is all about, right?} So far, many theses, some books, and several other publications have been typeset successfully with it. If you are interested in some typographic details behind it, enjoy Robert Bringhurst's wonderful book. % \citep{bringhurst:2002}.
%
%\paragraph{Important Note:} Some things of this style might look unusual at first glance, many people feel so in the beginning. However, all things are intentionally designed to be as they are, especially these:
%\begin{itemize}
%\item No bold fonts are used. Italics or spaced small caps do the job quite well.
%\item The size of the text body is intentionally shaped like it is. It supports both legibility and allows a reasonable amount of information to be on a page. And, no: the lines are not too short.
%\item The tables intentionally do not use vertical or double rules. See the documentation for the \texttt{booktabs} package for a nice discussion of this topic.\footnote{To be found online at \\ \url{http://www.ctan.org/tex-archive/macros/latex/contrib/booktabs/}.}
%\item And last but not least, to provide the reader with a way easier access to page numbers in the table of contents, the page numbers are right behind the titles. Yes, they are \emph{not} neatly aligned at the right side and they are \emph{not} connected with dots that help the eye to bridge a distance that is not necessary. If you are still not convinced: is your reader interested in the page number or does she want to sum the numbers up?
%\end{itemize}
%
%\noindent Therefore, please do not break the beauty of the style by changing these things unless you really know what you are doing! Please.
%
%%----------------------------------------------------------------------------------------
%
%\section{Organization}
%A very important factor for successful thesis writing is the organization of the material. This template suggests a structure as the following:
%\begin{itemize}
%\marginpar{You can use these margins for summaries of the text body\dots}
%\item\texttt{Chapters/} is where all the ``real'' content goes in separate files such as \texttt{Chapter01.tex} etc.
%\item\texttt{FrontBackMatter/} is where all the stuff goes that surrounds the ``real'' content, such as the acknowledgments, dedication, etc.
%\item\texttt{gfx/} is where you put all the graphics you use in the thesis. Maybe they should be organized into subfolders depending on the chapter they are used in, if you have a lot of graphics.
%\item\texttt{Bibliography.bib}: the Bib\TeX\ database to organize all the references you might want to cite.
%\item\texttt{classicthesis.sty}: the style definition to get this awesome look and feel. Bonus: works with both \LaTeX\ and \textsc{pdf}\LaTeX\dots and \mLyX.
%\item\texttt{ClassicThesis.tcp} a \TeX nicCenter project file. Great tool and it's free!
%\item\texttt{ClassicThesis.tex}: the main file of your thesis where all the content gets bundled together.
%\item\texttt{classicthesis-config.tex}: a central place to load all nifty packages that are used. In there, you can also activate backrefs in order to have information in the bibliography about where a source was cited in the text (\ie, the page number).
%    
%\emph{Make your changes and adjustments here.} This means that you specify here the options you want to load \texttt{classicthesis.sty} with. You also adjust the title of your thesis, your name, and all similar information here. Refer to \autoref{sec:custom} for more information.
%
%This had to change as of version 3.0 in order to enable an easy transition from the ``basic'' style to \mLyX.
%\end{itemize}
%
%\noindent In total, this should get you started in no time.
%
%%----------------------------------------------------------------------------------------
%
%\section{Style Options}\label{sec:options}
%
%There are a couple of options for \texttt{classicthesis.sty} that allow for a bit of freedom concerning the layout: \marginpar{\dots or your supervisor might use the margins for some comments of her own while reading.}
%\begin{itemize}
%\item General:
%\begin{itemize}
%\item\texttt{drafting}: prints the date and time at the bottom of each page, so you always know which version you are dealing with. Might come in handy not to give your Prof. that old draft.
%\end{itemize}
%	
%\item Parts and Chapters:
%\begin{itemize}
%\item\texttt{parts}: if you use Part divisions for your document, you should choose this option. (Cannot be used together with \texttt{nochapters}.)
%
%\item\texttt{nochapters}: allows to use the look-and-feel with classes that do not use chapters, \eg, for articles. Automatically turns off a couple of other options: \texttt{eulerchapternumbers}, \texttt{linedheaders}, \texttt{listsseparated}, and \texttt{parts}. 
%
%\item\texttt{linedheaders}: changes the look of the chapter headings a bit by adding a horizontal line above the chapter title. The chapter number will also be moved to the top of the page, above the chapter title.
%\end{itemize}
%
%\item Typography:
%\begin{itemize}
%\item\texttt{eulerchapternumbers}: use figures from Hermann Zapf's Euler math font for the chapter numbers. By default, old style figures from the Palatino font are used.
%
%\item\texttt{beramono}: loads Bera Mono as typewriter font. (Default setting is using the standard CM typewriter font.)
%\item\texttt{eulermath}: loads the awesome Euler fonts for math. (Palatino is used as default font.)
%
%\item\texttt{pdfspacing}: makes use of pdftex' letter spacing capabilities via the \texttt{microtype} package.\footnote{Use \texttt{microtype}'s \texttt{DVIoutput} option to generate DVI with pdftex.} This fixes some serious issues regarding math formul\ae\ etc. (\eg, ``\ss'') in headers. 
%
%\item\texttt{minionprospacing}: uses the internal \texttt{textssc} command of the \texttt{MinionPro} package for letter spacing. This automatically enables the \texttt{minionpro} option and overrides the \texttt{pdfspacing} option.
%\end{itemize}  
%
%\item Table of Contents:
%\begin{itemize}
%\item\texttt{tocaligned}: aligns the whole table of contents on the left side. Some people like that, some don't.
%
%\item\texttt{dottedtoc}: sets pagenumbers flushed right in the table of contents.
%
%\item\texttt{manychapters}: if you need more than nine chapters for your document, you might not be happy with the spacing between the chapter number and the chapter title in the Table of Contents. This option allows for additional space in this context. However, it does not look as ``perfect'' if you use \verb|\parts| for structuring your document.
%\end{itemize}
%
%\item Floats:
%\begin{itemize}
%\item\texttt{listings}: loads the \texttt{listings} package (if not already done) and configures the List of Listings accordingly.
%    
%\item\texttt{floatperchapter}: activates numbering per chapter for all floats such as figures, tables, and listings (if used).	
%    
%\item\texttt{subfig}(\texttt{ure}): is passed to the \texttt{tocloft} package to enable compatibility with the \texttt{subfig}(\texttt{ure}) package. Use this option if you want use classicthesis with the \texttt{subfig} package.
%
%\end{itemize}    
%
%\end{itemize}
%
%\noindent The best way to figure these options out is to try the different possibilities and see, what you and your supervisor like best.
%
%In order to make things easier in general, \texttt{classicthesis-config.tex} contains some useful commands that might help you.
%
%%----------------------------------------------------------------------------------------
%
%\section{Customization}\label{sec:custom}
%
%This section will give you some hints about how to adapt \texttt{classicthesis} to your needs.
%
%The file \texttt{classicthesis.sty} contains the core functionality of the style and in most cases will be left intact, whereas the file \texttt{classic\-thesis-config.tex} is used for some common user customizations. 
%
%The first customization you are about to make is to alter the document title, author name, and other thesis details. In order to do this, replace the data in the following lines of \texttt{classicthesis-config.tex:}\marginpar{Modifications in \texttt{classic\-thesis-config.tex}
%}
%
%\begin{lstlisting}[frame=lt]
%\newcommand{\myTitle}{A Classic Thesis Style\xspace}
%\newcommand{\mySubtitle}{An Homage to ...\xspace}
%\newcommand{\myDegree}{Doktor-Ingenieur (Dr.-Ing.)\xspace}
%\end{lstlisting}
%
%Further customization can be made in \texttt{classicthesis-config.tex} by choosing the options to \texttt{classicthesis.sty} (see~\autoref{sec:options}) in a line that looks like this:
%
%\begin{lstlisting}[frame=lt]
%\PassOptionsToPackage{eulerchapternumbers,listings,drafting, pdfspacing, subfig,beramono,eulermath,parts}{classicthesis}
%
%\end{lstlisting}
%
%If you want to use backreferences from your citations to the pages they were cited on, change the following line from:
%\begin{lstlisting}[breaklines=false,frame=lt]
%\setboolean{enable-backrefs}{false}
%\end{lstlisting}
%to
%\begin{lstlisting}[breaklines=false,frame=lt]
%\setboolean{enable-backrefs}{true}
%\end{lstlisting}
%
%Many other customizations in \texttt{classicthesis-config.tex} are possible, but you should be careful making changes there, since some changes could cause errors.
%
%Finally, changes can be made in the file \texttt{classicthesis.sty}, \marginpar{Modifications in \texttt{classicthesis.sty}} although this is mostly not designed for user customization. The main change that might be made here is the text-block size, for example, to get longer lines of text.
%
%%----------------------------------------------------------------------------------------
%
%\section{Issues}\label{sec:issues}
%This section will list some information about problems using \texttt{classic\-thesis} in general or using it with other packages.
%
%Beta versions of \texttt{classicthesis} can be found at the following Google code repository:
%\begin{center}
%\url{http://code.google.com/p/classicthesis/}
%\end{center}
%
%\noindent There, you can also post serious bugs and problems you encounter.
%
%\subsection*{Compatibility with the \texttt{glossaries} Package}
%If you want to use the \texttt{glossaries} package, take care of loading it with the following options:
%\begin{verbatim}
%\usepackage[style=long,nolist]{glossaries}
%\end{verbatim}
%
%\noindent Thanks to Sven Staehs for this information. 
%
%\subsection*{Compatibility with the (Spanish) \texttt{babel} Package}
%Spanish languages need an extra option in order to work with this template:
%\begin{verbatim}
%\usepackage[spanish,es-lcroman]{babel}
%\end{verbatim}
%
%\noindent Thanks to an unknown person for this information (via Google Code issue reporting). 
%
%\subsection*{Compatibility with the \texttt{pdfsync} Package}
%Using the \texttt{pdfsync} package leads to linebreaking problems with the \texttt{graffito} command. Thanks to Henrik Schumacher for this information. 
%
%%----------------------------------------------------------------------------------------
%
%\section{Future Work}
%So far, this is a quite stable version that served a couple of people well during their thesis time. However, some things are still not as they should be. Proper documentation in the standard format is still missing. In the long run, the style should probably be published separately, with the template bundle being only an application of the style. Alas, there is no time for that at the moment\dots it could be a nice task for a small group of \LaTeX nicians.
%
%Please do not send me email with questions concerning \LaTeX\ or the template, as I do not have time for an answer. But if you have comments, suggestions, or improvements for the style or the template in general, do not hesitate to write them on that postcard of yours.
%
%%----------------------------------------------------------------------------------------
%
%\section{License}
%\paragraph{GNU General Public License:} This program is free software; you can redistribute it and/or modify it under the terms of the \textsmaller{GNU} General Public License as published by the Free Software Foundation; either version 2 of the License, or (at your option) any later version.
%
%This program is distributed in the hope that it will be useful, but \emph{without any warranty}; without even the implied warranty of \emph{merchantability} or \emph{fitness for a particular purpose}. See the \textsmaller{GNU} General Public License for more details.